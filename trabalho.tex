\documentclass[
	oldfontcommands,
	% -- opções de customização --
	sumario=tradicional,
	%sumario=abnt-6027-2012,
	% -- opções da classe memoir --
	12pt,      % tamanho da fonte
	openright, % capítulos começam em pág ímpar (insere página vazia caso preciso)
	oneside,   % impressão em anverso (norma CCPG).
	a4paper,   % tamanho do papel. 
	% -- opções do pacote babel --
	english, % idioma adicional para hifenização
	brazil   % o último idioma é o principal do documento
	]{imecc-unicamp}

% configs
	% pacotes
	% pacotes básicos (essenciais ao modelo). NÃO RETIRAR NENHUM
	\usepackage{lmodern}            % Usa a fonte Latin Modern
	\usepackage[T1]{fontenc}        % Seleção de códigos de fonte.
	\usepackage[utf8]{inputenc}     % Codificação do documento (conversão automática dos acentos)
	\usepackage{lastpage}           % Usado pela Ficha catalográfica
	\usepackage{indentfirst}        % Indenta o primeiro parágrafo de cada seção.
	\usepackage{color,xcolor}       % Controle das cores
	\usepackage{graphicx}           % Inclusão de gráficos
	\usepackage{microtype}          % para melhorias de justificação
	\usepackage{hyperref}           % Amplo suporte para hipertexto em LaTeX
	\usepackage[brazilian]{backref} % Paginas com as citações na bibl
	\usepackage[alf,
			abnt-repeated-author-omit=yes,
			abnt-etal-list=0]{abntex2cite} % Citações padrão ABNT
	
% pacotes sugeridos
	\usepackage{lipsum}   % Textos de teste
	\usepackage{enumitem} % Auxiliar dos ambientes itemize, enumerate e description
	\usepackage{amsfonts} % Fontes AMS
	\usepackage{amsmath}  % Facilidades matemáticas
	\usepackage{tikz}     % Para desenhos e diagramas
	\usepackage{bm}       % Para a função \boldsymbol{} que escreve em negrito no modo matemático
	\usepackage{icomma}   % Remove os espaços após a vírgula no modo matemático (a menos que especificado)

	% \usepackage{verbatim}
	% \usepackage{float}
	% \usepackage{amsbsy}   % Para símbolos matemáticos em negrito
	% \usepackage{amscd}    % Para diagramas
	% \usepackage{amssymb}  % Para os símbolos mais antigos
	% \usepackage{amstext}  % Para fragmentos tipo texto em modo matemático
	% \usepackage{amsthm}   % Para teoremas
	% \usepackage{cleveref} % Referência cruzada inteligente
	% \usepackage{dsfont}   % Para o estilo de conjuntos de números $\mathds{R}$
	% \usepackage{ifthen}   % Comandos de condição em LaTeX
	% \usepackage{listings} % Para inserir códigos de outras linguagens de programação
	% \usepackage{lscape}   % Para imprimir alguma página no formato paisagem
	% \usepackage{mathabx}  % Conjunto de símbolos matemáticos
	% \usepackage{mathrsfs} % Suporte para fontes RSFS
	% \usepackage{pdfpages} % Para inserir páginas PDF no texto
	% \usepackage{subfig}   % Para figuras lado-a-lado
	% \usepackage[english,onelanguage]{algorithm2e} % Para inserir algoritmos (longend,vlined)

	
	% informações e dados para capa e folha de rosto
	% Informações gerais do trabalho EM LÍNGUA PORTUGUESA
	\titulo{Título da sua Dissertação de Mestrado ou Tese de Doutorado}
	\tipotrabalho{Dissertação} % ou Tese
	\curso{Estatística} % ou Matemática ou Matemática Aplicada ou Matemática Aplicada e Computacional
	% \curso{}  % Se for aluno do PROFMAT
	
% Suas informações NA LÍNGUA DO TRABALHO
	% Aluno do sexo MASCULINO:
	% \autor{Nome Completo do Aluno}
	% \titulacao{Mestre} % ou Doutor
	% Aluna do sexo FEMININO
	\autor[autora]{Nome Completo da Aluna}
	\titulacao{Mestra} % ou Doutora
	% orientador/coorientador do sexo MASCULINO
	% \orientador{Nome Completo do Orientador}
	% \coorientador{Nome Completo do Coorientador}
	% orientador/coorientador do sexo FEMININO (exceto se o trabalho for em inglês)
	\orientador[Orientadora]{Nome Completo da Orientadora}
	\coorientador[Coorientadora]{Nome Completo da Coorientadora}
	
% Data
	\data{Ano da banca}

	
	% configurações gerais
	% As configurações gerais são colocadas aqui, como novos comandos para o corpo do texto,
% informações de bookmark para o PDF, tamanho de parágrafos, entre outros.

% Formatação em geral
	% Formataçao da página
	\setlength{\parindent}{2.0cm} % O tamanho do parágrafo
	\setlength{\parskip}{0.2cm}  % tente também \onelineskip. Controle do espaçamento entre um parágrafo e outro
	
	% Formatação dos nomes nas seções
	% \chapterstyle{default} % Para que apareça o nome 'Capítulo X' antes do título de cada capítulo
	
	% Formatação do modo matemático
	\everymath{\displaystyle} % Com isso você não precisa se preocupar com a localização dos limitantes dos somatórios etc ($$\sum_{i = 1}^{n}$$)
	
	% cores
	\definecolor{blue}{RGB}{41,5,195}
	\definecolor{verde}{rgb}{0,0.5,0}
	
	% cores adicionais
	% Na minha dissertação eu optei por usar essas duas paletas de cores em várias situações sinta-se livre para modificar ou usar.
	% Pelo que me lembro, usei como base o site colorbrewer2.org
	\definecolor{pal1.1}{HTML}{8DA0CB}
	\definecolor{pal1.2}{HTML}{66C2A5}
	\definecolor{pal1.3}{HTML}{FC8D62}
	
	\definecolor{pal2.1}{HTML}{e41a1c}
	\definecolor{pal2.2}{HTML}{ff7f00}
	\definecolor{pal2.3}{HTML}{984ea3}
	\definecolor{pal2.4}{HTML}{377eb8}
	\definecolor{pal2.5}{HTML}{4daf4a}

% Novos comandos e ambientes
	% para facilitar com os pacotes
	\newcommand{\citep}{\citeonline} % Na minha dissertação usei o padrção de referências Nome Do Fulano (1999). Com esse ajuste você pode usar só o \citep{apelido_do_artigo}
	
	% Tikz - Usei muito esse exemplo de boxplot aqui nas legendas. Você pode apagar, modificar, etc.
	\newcommand{\boxplotLegenda}[1]{\tikz[scale=0.2, baseline = {(0ex, -0.5ex)}]{
		\filldraw[#1, draw = black] (0, -.5) rectangle +(1, 1.3);
		\draw[very thick] (0, -.1) -- +(1, 0);
		\draw (.5,-.5) -- +(0, -.2);
		\draw (.25,-.7) -- +(.5, 0);
		\draw (.5,.8) -- +(0, .4);
		\draw (.25,1.2) -- +(.5, 0);}}
	
	% Ambientes personalizados
	% \theoremstyle{plain}
	\newtheorem{theorem}{Teorema}%[chapter]
	\newtheorem{corollary}{Corolário}%[chapter]
	\providecommand*{\corollaryautorefname}{Corolário}
	\newtheorem{conjecture}{Conjectura}%[chapter]
	\providecommand*{\definitionautorefname}{Definição}
	\newtheorem{example}{Exemplo}%[chapter]
	\providecommand*{\exampleautorefname}{Exemplo}
	
	% Operadores abreviados
	\DeclareMathOperator{\argmin}{\mathrm{arg}\min}
	\DeclareMathOperator{\argmax}{\mathrm{arg}\max}
	\DeclareMathOperator{\sgn}{\mathrm{sgn}}
	\renewcommand{\sin}{\mathrm{sen}}
	\renewcommand{\tan}{\mathrm{tg}}
	\renewcommand{\csc}{\mathrm{cossec}}
	\renewcommand{\cot}{\mathrm{cotg}}

% Configurações de pacores
	% backref (recomendo não mexer)
		% Usado sem a opção hyperpageref de backref
		\renewcommand{\backrefpagesname}{Citado na(s) página(s):~}
		% Texto padrão antes do número das páginas
		\renewcommand{\backref}{}
		% Define os textos da citação
		\renewcommand*{\backrefalt}[4]{
			\ifcase #1 %
				Nenhuma citação no texto.%
			\or
				Citado na página #2.%
			\else
				Citado #1 vezes nas páginas #2.%
			\fi}

	% chngcntr
		\counterwithin{figure}{chapter} % Figuras enumeradas segundo o capítulo
		\counterwithin{table}{chapter} % Tabelas enumeradas segundo o capítulo

	% listings
		%\lstset{
		%	language=C++,
		%	basicstyle=\ttfamily, 
		%	keywordstyle=\color{blue}, 
		%	stringstyle=\color{verde}, 
		%	commentstyle=\color{red}, 
		%	extendedchars=true, 
		%	showspaces=false, 
		%	showstringspaces=false,
		%	numbers=left,
		%	numberstyle=\tiny,
		%	breaklines=true, 
		%	backgroundcolor=\color{green!10},
		%	breakautoindent=true,
		%	fontadjust=false
		%}
		
	% hyperref (recomendo não mexer)
		\makeatletter
		\hypersetup{
			%pagebackref=true,
			pdftitle={\@title},
			pdfauthor={\@author},
			pdfsubject={%
				\imprimirtipotrabalho\ apresentada ao Instituto de Matemática, Estatística %
				e Computação Científica da Universidade Estadual de Campinas como parte dos %
				requisitos exigidos para a obtenção do título de \imprimirtitulacao\ em %
				\imprimircurso.
			},
			pdfcreator={LaTeX with unicamp-abnTeX2},
			pdfkeywords={abnt}{latex}{abntex}{abntex2}{trabalho acadêmico},
			colorlinks=true,   % false: boxed links; true: colored links
			linkcolor=blue,    % color of internal links
			citecolor=blue,    % color of links to bibliography
			filecolor=magenta, % color of file links
			urlcolor=blue,     % color of internet links
			bookmarksdepth=4
		}
		\makeatother
% Gerenciamento de arquivos
	% Gerenciamento de pastas
	\graphicspath{{figuras/}} % Faz com que você não precise se referir à pasta figuras quando chamar no \includegraphics.

% ?
	\newsubfloat{figure} % Allow subfloats in figure environment
	\providecommand*{\subfigureautorefname}{Subfigura}

	
\begin{document}
	\selectlanguage{brazil} % Seleciona o idioma do documento (conforme pacotes do babel)
	\frenchspacing % Retira espaço extra obsoleto entre as frases.
	
	% Pré-textuais
		\pretextual
		
		\imprimirprimeirafolha % primeira folha (obrigatório)
		
		% folha de rosto (obrigatório)
			\imprimirfolhaderosto
		
		% ficha catalográfica (obrigatório)
			% A biblioteca da UNICAMP lhe fornecerá um PDF com a ficha catalográfica definitiva após a defesa
			% do trabalho {http://hamal.bc.unicamp.br/catalogonline2/}. Quando estiver com o documento, salve-o
			% como PDF no diretório 01-pretextuais e deixe apenas o comando '\includepdf{01-pretextuais/ficha-catalografica.pdf}'
			% dentro do ambiente abaixo
			\begin{fichacatalografica}
				\begin{center}
					{\ABNTEXchapterfont\large A ficha catalográfica deverá ser solicitada online via \url{http://www.sbu.unicamp.br/sbu/elaboracao-de-ficha-catalografica/}}
				\end{center}
				%\includepdf{01-pretextuais/ficha-catalografica.pdf}
			\end{fichacatalografica}
		
		% folha de aprovação (obrigatório)
			% A folha de aprovação será fornecida pela secretaria de pós-graduação. Após recebê-la, escaneie a folha
			% salvando em PDF no diretório 01-pretextuais com o nome 'folhadeaprovacao.pdf' e deixe apenas o comando
			% '\includepdf{01-pretextuais/folhadeaprovacao.pdf}' dentro do ambiente abaixo
			\begin{folhadeaprovacao}
				\centering{\ABNTEXchapterfont\large A folha de aprovação será fornecida pela Secretaria de Pós-Graduação}
				%\includepdf{01-pretextuais/folhadeaprovacao.pdf}
			\end{folhadeaprovacao}
		
		% dedicatória (opcional)
			\begin{dedicatoria}
				\vspace*{\fill}
				\centering
				\noindent
				\textit{
					Este trabalho é dedicado às crianças adultas que,\\
					quando pequenas, sonharam em se tornar cientistas.
				}
				\vspace*{\fill}
			\end{dedicatoria}
		
		% agradecimentos (opcional, mas a gratidão é uma virtude...)
			\begin{agradecimentos}
				Inserir aqui os agradecimentos. ATENÇÃO ALUNO BOLSISTA: você precisa mencionar agradecimento ao órgão de fomento de sua bolsa, incluindo explicitamente o código do processo. Caso seja bolsista CAPES, especial atenção ao Art. 3º da Portaria CAPES 206/2018, que prevê texto padrão de agradecimento. Caso seja bolsista CNPq, insira o nome completo do Conselho Nacional de Desenvolvimento Científico e Tecnológico.
			\end{agradecimentos}
			
		% epígrafe (opcional)
			\begin{epigrafe}
				\vspace*{\fill}
				\begin{flushright}
				\textit{``Passarinho? \\
					Que som é esse???? \\
					(Passarinho, Castelo Ra-tim-bum)
				}
				\end{flushright}
			\end{epigrafe}
		
		% resumos (obrigatório)
			% -------------------------------------------------------------
%  RESUMOS
\setlength{\absparsep}{18pt} % ajusta o espaçamento dos parágrafos do resumo
% -------------------------------------------------------------
% ATENÇÃO: o ambiente 'otherlanguage*' deve ser usado para o resumo que não está na
% língua vernácula do trabalho, com a respectiva opção linguística do pacote 'babel'.
% -------------------------------------------------------------
% resumo em PORTUGUÊS (OBRIGATÓRIO)
\begin{resumo}[Resumo]
 \begin{otherlanguage*}{brazil}
    Segundo a \citeonline[3.1-3.2]{NBR6028:2003}, o resumo deve ressaltar o objetivo,
    o método, os resultados e as conclusões do documento. A ordem e a extensão destes
    itens dependem do tipo de resumo (informativo ou indicativo) e do tratamento que
    cada item recebe no documento original. O resumo deve ser precedido da referência
    do documento, com exceção do resumo inserido no próprio documento. Neste trabalho,
    devem ser utilizadas até 500 palavras. (\ldots) As palavras-chave devem figurar
    logo abaixo do resumo, antecedidas da expressão Palavras-chave:, separadas entre
    si por ponto e finalizadas também por ponto.

    \textbf{Palavras-chave}: Latex. Abntex. Editoração de texto.
 \end{otherlanguage*}
\end{resumo}
% -------------------------------------------------------------
% -------------------------------------------------------------
% resumo em INGLÊS (OBRIGATÓRIO)
\begin{resumo}[Abstract]
 \begin{otherlanguage*}{english}
    This is the english abstract.
    
    \textbf{Keywords}: Latex. Abntex. Text editoration.
 \end{otherlanguage*}
\end{resumo}
% -------------------------------------------------------------
		
		% lista de ilustrações (opcional)
			%\pdfbookmark[0]{\listfigurename}{lof}
			%\listoffigures*
			%\cleardoublepage
		
		% lista de tabelas (opcional)
			%\pdfbookmark[0]{\listtablename}{lot}
			%\listoftables*
			%\cleardoublepage
		
		% lista de abreviaturas e siglas (opcional)
			%\begin{siglas}
			%	\item[UNICAMP] Universidade Estadual de Campinas
			%	\item[IMECC] Instituto de Matemática, Estatística e Computação Científica
			%\end{siglas}
		
		% lista de símbolos (opcional)
			%\begin{simbolos}
			%	\item[$\nabla f$] Gradiente da função $f$
			%	\item[$\infty$] Infinito
			%\end{simbolos}
			
		% lista de algoritmos (opcional)
			%\pdfbookmark[0]{\listalgorithmcfname}{loa}
			%\listofalgorithms
			%\cleardoublepage
			
		% lista de códigos (opcional)
			%\pdfbookmark[0]{\lstlistlistingname}{lol}
			%\begin{KeepFromToc}
			%\lstlistoflistings
			%\end{KeepFromToc}
			%\cleardoublepage
			
		% sumário (obrigatório)
			\pdfbookmark[0]{\contentsname}{toc}
			\tableofcontents*
			\cleardoublepage
		
	% Textuais
		\textual
		
		% introdução
		\chapter{Introdução} \label{cap_intro}
		% ----------------------------------------------------------
% Exemplo de capítulo sem numeração, mas presente no Sumário
\chapter*[Introdução]{Introdução}
\addcontentsline{toc}{chapter}{Introdução}
% ----------------------------------------------------------

Este documento e seu código-fonte são exemplos de referência de uso da classe
\textsf{abntex2} e do pacote \textsf{abntex2cite}. O documento exemplifica a elaboração 
de trabalho acadêmico (teses e dissertações) produzido conforme a \textbf{Informação 
CCPG/001/2015} (que trata das \emph{Normas para impressão de teses/dissertações} da 
UNICAMP). Encorajamos o leitor a consultar a Informação CCPG/001/2015 \cite{CCPG:001:2015}
antes de iniciar as alterações neste documento e seu código-fonte.

A elaboração deste modelo teve como base uma customização do ``Modelo Canônico de
Trabalho Acadêmico com \abnTeX'' \cite{abntex2modelo} para que as normas presentes na
Informação CCPG/001/2015 fossem respeitadas. O modelo original produzido pela equipe 
\abnTeX\ cumpre as seguintes normas ABNT:
\begin{enumerate}
 \item \textbf{ABNT NBR 14724:2011}: Informação e documentação - Trabalhos 
    acadêmicos - Apresentação;
 \item \textbf{ABNT NBR 10520:2002}: Informação e documentação - Citações;
 \item \textbf{ABNT NBR 6034:2004}: Informação e documentação - Índice - Apresentação;
 \item \textbf{ABNT NBR 6028:2003}: Informação e documentação - Resumo - Apresentação;
 \item \textbf{ABNT NBR 6027:2012}: Informação e documentação - Sumário - Apresentação;
 \item \textbf{ABNT NBR 6024:2012}: Informação e documentação - Numeração progressiva 
    das seções de um documento - Apresentação
 \item \textbf{ABNT NBR 6023:2002}: Informação e documentação - Referência - Elaboração.
\end{enumerate}

Este documento deve ser utilizado como complemento dos manuais do \abnTeX\ 
\cite{abntex2classe,abntex2cite,abntex2cite-alf} e da classe \textsf{memoir} \cite{memoir}.

A leitura do teor desde documento (tanto o PDF quando os arquivos que compõem seu código-fonte),
bem como do arquivo \textsf{LEIAME.txt} é altamente recomendada para melhor entendimento da 
dinâmica de funcionamento da classe \textsf{abntex2} e do pacote \textsf{abntex2cite}. Seus
principais comandos e usos estão exemplificados no decorrer do texto, bem como outras informações
relevantes para a escrita de seu trabalho acadêmico.
		
		% desenvolvimento
		\chapter{Nome_do_cap_1} \label{cap_ideia_do_nome}
		Você não precisa definir capítulos aqui. Na minha dissertação eu só usei capítulos no arquivo principal e este aqui ficou com

\section{Seções}
	como essa 
	
	\subsection{subseções}
		como essa, e 
		
		\subsubsection{subsubseções (muito raramente)}
			como essa
			
\section{Além disso}
	Eu sugiro que você salve as suas figuras e tabelas nas pastas de figuras e tabelas. Pode dividir essas pastas com base nos capítulos se quiser, mas deixe essas coisas separadas para não bagunçar o resto do template (isso também ajuda MUITO na hora de preparar a sua apresentação para a banca...).
 % Você precisa criar o arquivo e editar essa linha
		
		% \chapter{Nome_do_cap_2}
		% \input{02-textuais/3-cap_2.tex} % Você precisa criar o arquivo e editar essa linha
		
		% \chapter{Nome_do_cap_3}
		% \input{02-textuais/4-cap_3.tex} % Você precisa criar o arquivo e editar essa linha
		
		% considerações finais
		% Finaliza a parte no bookmark do PDF para que se inicie o
		% bookmark na raiz e adiciona espaço de parte no Sumário
		% \phantompart
		% \input{02-textuais/ZZ-considfinais.tex}

	% pós-textuais
		\postextual

	% referências
		%\nocite{masolo2010}
		%\bibliography{$BIBLI}
		% VOCÊ TEM QUE ADICIONAR UM ARQUIVO AQUI. Se trabalha em localmente linux, recomendo criar um arquivo .bib único para o seu sistema e setar a variável global BIBLI para esse arquivo.

	% apêndices
	%\begin{apendicesenv} % Imprime uma página indicando o início dos apêndices
	%	\partapendices
	%	\chapter{Quisque libero justo}

\lipsum[50]

\chapter{Nullam elementum urna vel imperdiet sodales elit ipsum pharetra ligula
ac pretium ante justo a nulla curabitur tristique arcu eu metus}

\lipsum[55-57]
	%\end{apendicesenv}
	
	% anexos
	\begin{anexosenv} % Imprime uma página indicando o início dos anexos
		\partanexos
		
		\chapter{Nome do Anexo I} \label{A_ideia_do_nome_do_anexoI}
		\section{Você também pode usar seções e subseções aqui!!}
	\lipsum

		
		%\chapter{Nome do Anexo II} \label{A_ideia_do_nome_do_anexoII}
		%\input{03-pós_textuais/02-anexo2.tex}
			
	\end{anexosenv}

\end{document}
