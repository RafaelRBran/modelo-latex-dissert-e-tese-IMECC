% As configurações gerais são colocadas aqui, como novos comandos para o corpo do texto,
% informações de bookmark para o PDF, tamanho de parágrafos, entre outros.

% Formatação em geral
	% Formataçao da página
	\setlength{\parindent}{2.0cm} % O tamanho do parágrafo
	\setlength{\parskip}{0.2cm}  % tente também \onelineskip. Controle do espaçamento entre um parágrafo e outro
	
	% Formatação dos nomes nas seções
	% \chapterstyle{default} % Para que apareça o nome 'Capítulo X' antes do título de cada capítulo
	
	% Formatação do modo matemático
	\everymath{\displaystyle} % Com isso você não precisa se preocupar com a localização dos limitantes dos somatórios etc ($$\sum_{i = 1}^{n}$$)
	
	% cores
	\definecolor{blue}{RGB}{41,5,195}
	\definecolor{verde}{rgb}{0,0.5,0}
	
	% cores adicionais
	% Na minha dissertação eu optei por usar essas duas paletas de cores em várias situações sinta-se livre para modificar ou usar.
	% Pelo que me lembro, usei como base o site colorbrewer2.org
	\definecolor{pal1.1}{HTML}{8DA0CB}
	\definecolor{pal1.2}{HTML}{66C2A5}
	\definecolor{pal1.3}{HTML}{FC8D62}
	
	\definecolor{pal2.1}{HTML}{e41a1c}
	\definecolor{pal2.2}{HTML}{ff7f00}
	\definecolor{pal2.3}{HTML}{984ea3}
	\definecolor{pal2.4}{HTML}{377eb8}
	\definecolor{pal2.5}{HTML}{4daf4a}

% Novos comandos e ambientes
	% para facilitar com os pacotes
	\newcommand{\citep}{\citeonline} % Na minha dissertação usei o padrção de referências Nome Do Fulano (1999). Com esse ajuste você pode usar só o \citep{apelido_do_artigo}
	
	% Tikz - Usei muito esse exemplo de boxplot aqui nas legendas. Você pode apagar, modificar, etc.
	\newcommand{\boxplotLegenda}[1]{\tikz[scale=0.2, baseline = {(0ex, -0.5ex)}]{
		\filldraw[#1, draw = black] (0, -.5) rectangle +(1, 1.3);
		\draw[very thick] (0, -.1) -- +(1, 0);
		\draw (.5,-.5) -- +(0, -.2);
		\draw (.25,-.7) -- +(.5, 0);
		\draw (.5,.8) -- +(0, .4);
		\draw (.25,1.2) -- +(.5, 0);}}
	
	% Ambientes personalizados
	% \theoremstyle{plain}
	\newtheorem{theorem}{Teorema}%[chapter]
	\newtheorem{corollary}{Corolário}%[chapter]
	\providecommand*{\corollaryautorefname}{Corolário}
	\newtheorem{conjecture}{Conjectura}%[chapter]
	\providecommand*{\definitionautorefname}{Definição}
	\newtheorem{example}{Exemplo}%[chapter]
	\providecommand*{\exampleautorefname}{Exemplo}
	
	% Operadores abreviados
	\DeclareMathOperator{\argmin}{\mathrm{arg}\min}
	\DeclareMathOperator{\argmax}{\mathrm{arg}\max}
	\DeclareMathOperator{\sgn}{\mathrm{sgn}}
	\renewcommand{\sin}{\mathrm{sen}}
	\renewcommand{\tan}{\mathrm{tg}}
	\renewcommand{\csc}{\mathrm{cossec}}
	\renewcommand{\cot}{\mathrm{cotg}}

% Configurações de pacores
	% backref (recomendo não mexer)
		% Usado sem a opção hyperpageref de backref
		\renewcommand{\backrefpagesname}{Citado na(s) página(s):~}
		% Texto padrão antes do número das páginas
		\renewcommand{\backref}{}
		% Define os textos da citação
		\renewcommand*{\backrefalt}[4]{
			\ifcase #1 %
				Nenhuma citação no texto.%
			\or
				Citado na página #2.%
			\else
				Citado #1 vezes nas páginas #2.%
			\fi}

	% chngcntr
		\counterwithin{figure}{chapter} % Figuras enumeradas segundo o capítulo
		\counterwithin{table}{chapter} % Tabelas enumeradas segundo o capítulo

	% listings
		%\lstset{
		%	language=C++,
		%	basicstyle=\ttfamily, 
		%	keywordstyle=\color{blue}, 
		%	stringstyle=\color{verde}, 
		%	commentstyle=\color{red}, 
		%	extendedchars=true, 
		%	showspaces=false, 
		%	showstringspaces=false,
		%	numbers=left,
		%	numberstyle=\tiny,
		%	breaklines=true, 
		%	backgroundcolor=\color{green!10},
		%	breakautoindent=true,
		%	fontadjust=false
		%}
		
	% hyperref (recomendo não mexer)
		\makeatletter
		\hypersetup{
			%pagebackref=true,
			pdftitle={\@title},
			pdfauthor={\@author},
			pdfsubject={%
				\imprimirtipotrabalho\ apresentada ao Instituto de Matemática, Estatística %
				e Computação Científica da Universidade Estadual de Campinas como parte dos %
				requisitos exigidos para a obtenção do título de \imprimirtitulacao\ em %
				\imprimircurso.
			},
			pdfcreator={LaTeX with unicamp-abnTeX2},
			pdfkeywords={abnt}{latex}{abntex}{abntex2}{trabalho acadêmico},
			colorlinks=true,   % false: boxed links; true: colored links
			linkcolor=blue,    % color of internal links
			citecolor=blue,    % color of links to bibliography
			filecolor=magenta, % color of file links
			urlcolor=blue,     % color of internet links
			bookmarksdepth=4
		}
		\makeatother
% Gerenciamento de arquivos
	% Gerenciamento de pastas
	\graphicspath{{figuras/}} % Faz com que você não precise se referir à pasta figuras quando chamar no \includegraphics.

% ?
	\newsubfloat{figure} % Allow subfloats in figure environment
	\providecommand*{\subfigureautorefname}{Subfigura}
