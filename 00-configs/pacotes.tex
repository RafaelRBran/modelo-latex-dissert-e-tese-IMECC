% pacotes básicos (essenciais ao modelo). NÃO RETIRAR NENHUM
	\usepackage{lmodern}            % Usa a fonte Latin Modern
	\usepackage[T1]{fontenc}        % Selecao de codigos de fonte.
	\usepackage[utf8]{inputenc}     % Codificacao do documento (conversão automática dos acentos)
	\usepackage{lastpage}           % Usado pela Ficha catalográfica
	\usepackage{indentfirst}        % Indenta o primeiro parágrafo de cada seção.
	\usepackage{color,xcolor}       % Controle das cores
	\usepackage{graphicx}           % Inclusão de gráficos
	\usepackage{microtype}          % para melhorias de justificação
	\usepackage{hyperref}           % Amplo suporte para hipertexto em LaTeX
	\usepackage[brazilian]{backref} % Paginas com as citações na bibl
	\usepackage[alf,
			abnt-repeated-author-omit=yes,
			abnt-etal-list=0]{abntex2cite} % Citações padrão ABNT
	
% pacotes sugeridos
	\usepackage{lipsum}   % textos de teste
	\usepackage{enumitem} % auxiliar dos ambientes itemize, enumerate e description
	\usepackage{amsfonts} % fontes AMS
	\usepackage{amsmath}  % facilidades matemáticas

	% \usepackage{verbatim}
	% \usepackage{amsbsy}   % para símbolos matemáticos em negrito
	% \usepackage{amscd}    % para diagramas
	% \usepackage{amssymb}  % para os símbolos mais antigos
	% \usepackage{amstext}  % para fragmentos tipo texto em modo matemático
	% \usepackage{amsthm}   % para teoremas
	% \usepackage{cleveref} % Referência cruzada inteligente
	% \usepackage{dsfont}   % para o estilo de conjuntos de números $\mathds{R}$
	% \usepackage{ifthen}   % comandos de condição em LaTeX
	% \usepackage{listings} % para inserir códigos de outras linguagens de programação
	% \usepackage{lscape}   % para imprimir alguma página no formato paisagem
	% \usepackage{mathabx}  % conjunto de simbolos matemáticos
	% \usepackage{mathrsfs} % suporte para fontes RSFS
	% \usepackage{pdfpages} % para inserir páginas PDF no texto
	% \usepackage[english,onelanguage]{algorithm2e} % para inserir algoritmos (longend,vlined)
